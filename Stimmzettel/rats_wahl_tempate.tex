\documentclass[a4paper,9pt]{scrartcl}

\usepackage[utf8]{inputenc}
\usepackage{ngerman}
\usepackage{latexsym}
\usepackage[top=0.5in,bottom=0.5in]{geometry}
\usepackage{graphicx}
\usepackage{url}

\usepackage{multicol}

\pagestyle{empty}
\setlength{\parindent}{0cm}
\begin{document}
\huge
\hfill
\includegraphics[scale=0.3]{wusel_mono.pdf}
\vspace{-1em}
\section*{Stimmzettel FSV-Wahl Informatik 2022}


\begin{list}{$\bigcirc$}{}
\item Jonas Wimmer
\item Lina Schuster
\item Philipp Gruber
\item Alina Vogel
\item Erik Bachmann
\item Mia Bergmann
\item Tim Neubert
\item Anna Hofer
\item Felix Sauer
\item Laura Engel

\vspace{0,3cm}
\item \hrulefill \hspace{3cm}

\end{list}
\vspace{2em}
\large
Du hast \underline{\textbf{1} Stimme}, die ohne Bindung an die vorgeschlagenen Kandidierenden vergeben werden kann.
Wählbar ist jede ins Wählendenverzeichnis eingetragene Person (bei Uneindeutigkeit bitte Matrikelnummer angeben).
Wird keine Stimme vergeben, zählt dies als Enthaltung.
Bitte kreuze die zu wählende Person deutlich an.

Wenn du vor der Stimmabgabe Einsicht in das Wählendenverzeichnis nehmen möchtest, frag bei den Wahlhelfenden an der Urne.

\vspace{1em}
You have \underline{\textbf{1} vote} that you can either give to one of the candidates above or any person eligible to vote.
If you do not mark any circle it will count as an abstention.
Please make sure the person you want to vote is clearly marked.

If you want to have access to the electoral register you can ask the election workers at the poll.

\end{document}
